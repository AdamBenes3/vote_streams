\documentclass[a4paper,12pt]{report}
\addtolength{\topmargin}{-0.8in}
\setlength\textheight{248mm}
\setlength\textwidth{165mm}
\setlength\oddsidemargin{-2mm}
\setlength\evensidemargin{-2mm}
\usepackage[czech]{babel}
\usepackage[a-2u]{pdfx}
\usepackage{graphicx}
\usepackage{caption}
\usepackage{lmodern,textcomp}
\usepackage[T1]{fontenc}
\usepackage{fancyhdr}
\usepackage{xcolor,colortbl}
\usepackage{graphicx}
\usepackage{amssymb}
\usepackage{pgfplots}
\usepackage{multicol}
\usepackage{ulem}
\usepackage{color}
\usepackage{amsmath, amsthm, amssymb}
\usepackage{setspace}
\usepackage{listings}
\usepackage{cancel}
\usepackage{marvosym}
\usepackage{multicol}

\makeatletter
\newcommand*{\rom}[1]{\expandafter\@slowromancap\romannumeral #1@}
\makeatother

\definecolor{dkgreen}{rgb}{0,0.6,0}
\definecolor{gray}{rgb}{0.5,0.5,0.5}
\definecolor{mauve}{rgb}{0.58,0,0.82}
\graphicspath{ {../images/} }
\def\columnseprulecolor{\color{black}}
\setlength{\columnseprule}{0.3pt}
\def\title{Specification}
\def\author{Adam Beneš}
\setlength{\parindent}{0pt}
\def\doubleunderline#1{\underline{\underline{#1}}}\makeatletter
\def\@makechapterhead#1{
	{\parindent \z@ \raggedright \normalfont
		\huge\bfseries \thechapter. #1
		\par\nobreak
		\vskip 20\p@
}}
\def\@makeschapterhead#1{
	{\parindent \z@ \raggedright \normalfont
		\huge\bfseries #1
		\par\nobreak
		\vskip 20\p@
}}
\makeatother

\renewcommand{\chaptermark}[1]{%
	\markboth{#1}{}}

\fancypagestyle{toc}{
	\fancyhf{}
	\renewcommand{\headrulewidth}{0.4pt}
	\renewcommand{\footrulewidth}{0.4pt}
	\fancyhead[C]{}
	\fancyhead[L]{\textbf{\title}}
	\fancyfoot[L]{\author}
	\fancyfoot[C]{}
	\fancyfoot[R]{\thepage}
}

\fancypagestyle{plain}{
	\fancyhf{}
	\renewcommand{\headrulewidth}{0.4pt}
	\renewcommand{\footrulewidth}{0.4pt}
	\fancyhead[C]{}
	\fancyhead[L]{\textbf{\title\ -- \thechapter. \leftmark}}
	\fancyfoot[L]{\author}
	\fancyfoot[C]{}
	\fancyfoot[R]{\thepage}
}


\definecolor{codegreen}{rgb}{0,0.6,0}
\definecolor{codegray}{rgb}{0.5,0.5,0.5}
\definecolor{codepurple}{rgb}{0.58,0,0.82}
\definecolor{backcolour}{rgb}{0.95,0.95,0.92}

\lstdefinestyle{mystyle}{
	backgroundcolor=\color{backcolour},   
	commentstyle=\color{codegreen},
	keywordstyle=\color{magenta},
	numberstyle=\tiny\color{codegray},
	stringstyle=\color{codepurple},
	basicstyle=\ttfamily\footnotesize,
	breakatwhitespace=false,         
	breaklines=true,                 
	captionpos=b,                    
	keepspaces=false,                 
	numbers=left,                    
	numbersep=5pt,                  
	showspaces=false,                
	showstringspaces=false,
	showtabs=false,                  
	tabsize=2
}

\lstset{style=mystyle}

\begin{document}
	\pagenumbering{roman}
	\addtocontents{toc}{\protect\thispagestyle{toc}}
	\pagestyle{toc}
	\tableofcontents
	\cleardoublepage
	\pagestyle{plain}
	\pagenumbering{arabic}
	\pagebreak
	
	\chapter{Úvod}
	
	\section{Hrubé zadání projektu}
	
	Vytvořit program na zpracování voleb s různými volebními pravidly, volby budou zpracovávány pomocí streaming algoritmů.
	
	\section{Kde projekt žije}
	
	Projekt je umístěný na tomto \href{https://gitlab.com/adam_b3n3s/vote_streams}{Gitlabu} a na tomto \href{https://github.com/AdamBenes3/vote_streams}{Githubu}.
	
	\chapter{Podrobnější specifikace}
	
	\section{Knihovny}
	
	\begin{itemize}
		\item \href{https://github.com/PrefLib}{Preflib}.
		\item ...
	\end{itemize}
	
	\section{Zdroje}
	
	\begin{itemize}
		\item \href{http://dimacs.rutgers.edu/~graham/ssbd.html}{Kniha o streaming algoritmech}.
		\item \href{https://dl.acm.org/doi/pdf/10.1145/198429.198435 }{Článek reservoir sampling}.
		\item \href{https://en.wikipedia.org/wiki/Reservoir_sampling}{Wiki reservoir sampling}.
		\item ...
	\end{itemize}
	
	\section{Algoritmy}
	
	\begin{itemize}
		\item Misra-Gries algoritmus.
		\item Reservoir sampling algoritmus.
		\item ...
	\end{itemize}
	
	\section{Dekompozice projektu}
	
	\subsection{Souborvý strom}
	
	\begin{itemize}
		\item src:
		
		Obsahuje veškeré zrojové kódy.
		
		\item docs:
		
		Obsahuje .tex soubory spjaté s projektem, specifikace, uživatelská dokumentace, vývojářská dokumentace. Vše alespoň v jazycích:
		
		\begin{itemize}
			\item Čeština.
			\item Angličtina.
		\end{itemize}
		
		\item votes:
		
		Obsahuje soubory s volebními lístky.
		
		\item test:
		
		Obsahuje testy funkcí v src.
	\end{itemize}
	
	\subsection{Dekompozice src}
	
	\subsubsection{main.py}
	
	Jedná se o hlavní metodu která bude číst data ze souboru, volat funkce z jiných Python souborů, které budou obsahovat jednotlivé voting stream algoritmy a další.
	
	\subsubsection{vote\_generator.py}
	
	Jedná se o třídu, která bude mít za úkol generovat volby, různými způsoby:
	
	\begin{itemize}
		\item Exponenciální rozdělení.
		\item Normální rozdělení (Gaussova křivka).
		\item ...
	\end{itemize}
	
	\subsection{Soubory jednotlivých volebních pravidel}
	
	V souboru vote\_rules. Obsahuje implementaci jednotlyvých volebních pravidel.
	
	\subsection{Sampling algoritmy}
	
	Výběr nějakého řádově menšího počtu lístků.
	
	
	
	
	
	
	
	
	
	
	
	
\end{document}